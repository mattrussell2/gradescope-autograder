\documentclass[11pt]{report}
\usepackage{color}
\usepackage{hyperref}
\usepackage{xcolor}
\usepackage{listings}
\usepackage{graphicx}
\usepackage{enumitem}
\pagestyle{headings}

% Required packages
\usepackage{color}
\usepackage{xcolor}
\usepackage{listings}
\usepackage{moresize}

% Solarized colour scheme for listings - thanks https://marcusmo.co.uk/blog/latex-syntax-highlighting/ !!
\definecolor{solarized@base03}{HTML}{002B36}
\definecolor{solarized@base02}{HTML}{073642}
\definecolor{solarized@base01}{HTML}{586e75}
\definecolor{solarized@base00}{HTML}{657b83}
\definecolor{solarized@base0}{HTML}{839496}
\definecolor{solarized@base1}{HTML}{93a1a1}
\definecolor{solarized@base2}{HTML}{EEE8D5}
\definecolor{solarized@base3}{HTML}{FDF6E3}
\definecolor{solarized@yellow}{HTML}{B58900}
\definecolor{solarized@orange}{HTML}{CB4B16}
\definecolor{solarized@red}{HTML}{DC322F}
\definecolor{solarized@magenta}{HTML}{D33682}
\definecolor{solarized@violet}{HTML}{6C71C4}
\definecolor{solarized@blue}{HTML}{268BD2}
\definecolor{solarized@cyan}{HTML}{2AA198}
\definecolor{solarized@green}{HTML}{859900}

% define c++ code blocks - 
\lstnewenvironment{codeblock}[1][]{\lstset{language=C++,
        basicstyle=\footnotesize\ttfamily,
        numbers=left,
        numberstyle=\footnotesize,
        tabsize=2,
        breaklines=true,
        escapeinside={@}{@},
        numberstyle=\tiny\color{solarized@base01},
        keywordstyle=\color{solarized@green},
        stringstyle=\color{solarized@cyan}\ttfamily,
        identifierstyle=\color{solarized@blue},
        commentstyle=\color{solarized@base01},
        emphstyle=\color{solarized@red},
        frame=single,
        rulecolor=\color{solarized@base2},
        rulesepcolor=\color{solarized@base2},
        showstringspaces=false,
        #1
    }}{}

\lstnewenvironment{bashnumberedcodeblock}{\lstset{language=Bash,
        basicstyle=\footnotesize\ttfamily,
        numbers=left,
        numberstyle=\footnotesize,
        tabsize=2,
        breaklines=true,
        escapeinside={@}{@},
        frame=single,
        showstringspaces=false,
    }}{}

\lstnewenvironment{bashcodeblock}[1][]{
    \lstset{language=Bash,
        basicstyle=\footnotesize\ttfamily,
        commentstyle=\color{solarized@base01}
        tabsize=2,
        breaklines=true,
        escapeinside={!}{!},
        frame=single,
        showstringspaces=false,
        moredelim=**[is][\color{blue}]{@@}{@@},
        moredelim=**[is][\color{red}]{@!}{@!},
        moredelim=**[is][\color{solarized@cyan}]{!!}{!!},
        #1
    }}{}

\lstnewenvironment{tinybashcodeblock}[1][]{
    \lstset{language=Bash,
        basicstyle=\ssmall\ttfamily,
        commentstyle=\color{solarized@base01}
        tabsize=2,
        breaklines=true,
        escapeinside={!}{!},
        frame=single,
        showstringspaces=false,
        moredelim=**[is][\color{blue}]{@@}{@@},
        moredelim=**[is][\color{red}]{@!}{@!},
        moredelim=**[is][\color{solarized@cyan}]{!!}{!!},
        #1
    }}{}

\newcommand{\codeinline}{\lstinline[language=C++,
        basicstyle=\normalsize\ttfamily,
        tabsize=2,
        numberstyle=\tiny\color{solarized@base01},
        keywordstyle=\color{solarized@green},
        stringstyle=\color{solarized@cyan}\ttfamily,
        identifierstyle=\color{solarized@blue},
        commentstyle=\color{solarized@base01},
        emphstyle=\color{solarized@red}%,
        %#1
    ]}


\newcommand{\code}{\lstinline}
%for pretty ~ in code
\lstset{
    literate={~} {$\sim$}{1}
}
\lstset{
    basicstyle=\ttfamily,
    showstringspaces=false,
    commentstyle=\color{red},
    escapeinside={!}{!},
}

\def\labelitemii{$\circ$}
\def\labelitemiii{-}



\begin{document}

\chapter*{CS 15 Homework 0: ArrayLists}
\includegraphics[scale=1.25]{cheshireCat.png}

\section*{Introduction}
In this assignment you will implement a version of the array list data
structure discussed in class that contains characters.

Recall that an array list is a kind of list: An ordered collection of
data values.  ``Ordered" here does not mean ``sorted," it just means that,
if there are items in the list, there is a distinct first element,
second element, etc.  We will use 0-based indexing, i. e., the first
element in an array list will be element 0.  Note that an array list
cannot have ``holes": if you remove the fifth element (element 4) from a
10-element array list, then there are 9 elements left, and their
positions are 0 through 8.

You will write both the public and private sections of the \code{CharArrayList}
class.  The class definition will go in a file named  \code{CharArrayList.h};
the class implementation will go in a file named \code{CharArrayList.cpp}.  You
will also write test code in \code{testCharArrayList.cpp} (described below).

We'll describe the (public) interface first, then give some some
implementation specifics, and finally submission instructions.

\section*{Program Specification}

\subsection*{Important Notes}
\fcolorbox{red}{white}{\parbox{32em}{
    \begin{itemize}
        \item The names of your functions / methods as well as the order and types of parameters and return types must be exactly as specified. This is important because we will be compiling the class you wrote with our own client code! 
        \item  Any exception messages should likewise print exactly as specified and use the given error type. 
        \item You may not have any other public functions. 
        \item All data members must be private. 
        \item You may \textbf{not} use any C++ strings in your \code{CharArrayList} implementation (except for exception messages). 
        \item You may \textbf{not} use \code{std::vector} or any other built-in facility that would render the assignment trivial.
    \end{itemize}
}}
\newline\newline

\subsection*{Interface}
Your class must have the following interface (all the following members are public):
\begin{itemize}
    \item Define the following constructors for the \code{CharArrayList} class: 
    \begin{itemize}
        \item The default constructor takes no parameters and initializes an
        empty array list.  This array list has an initial capacity of 0.
        \item The second constructor takes in a single character as a parameter
        and creates a one element array list consisting of that character.
        This array list should have an initial capacity of 1.
        \item The third constructor takes an array of characters and the integer
        length of that array of characters as parameters.  It will create
        an array list containing the characters in the array.  This array
        list should have an initial capacity equal to the length of the
        array of characters that was passed.
        \item A copy constructor for the class that makes a deep copy of a given
        instance.
    \end{itemize}
     Recall that all constructors have no return type.
     \item  Define a destructor that destroys/deletes/recycles all
      heap-allocated data in the current array list.  It has no
      parameters and returns nothing.

    \item Define an assignment operator for the class that recycles the
      storage associated with the instance on the left of the assignment
      and makes a deep copy of the instance on the right hand side into
      the instance on the left hand side.

    \item An \code{isEmpty} function that takes no parameters and returns a
      boolean value that is true if this specific instance of the class
      is empty (has no characters) and false otherwise.

    \item A \code{clear} function that takes no parameters and has a \code{void} return
      type.  It makes the instance into an empty array list.  For
      example if you call the \code{clear} function and then the \code{isEmpty}
      function the \code{isEmpty} function should return true.

    \item A \code{size} function that takes no parameters and returns an integer
      value that is the number of characters in the array list.  The
      size of an array list is 0 if and only if it isEmpty.
     
     \item A \code{first} function that takes no parameters and returns the first
      element \code{(char)} in the array list.  If the array list is empty it
      should throw a \code{std::runtime_error} exception with the message
      ``cannot get first of empty ArrayList".

    \item A \code{last} function that takes no parameters and returns the last
      element \code{(char)} in the array list.  If the array list is empty it
      throws a \code{std::runtime_error} exception with the message ``cannot
      get last of empty ArrayList".

    \item An \code{elementAt} function that takes an integer index and returns
      the element \code{(char)} in the array list at that index.  NOTE: Indices
      are 0-based.  If the index is out of range it should throw a C++
      \code{std::range_error} exception with the message ``index (IDX) not in range
      [0..SIZE)" where IDX is the index that was given and SIZE is the
      size of the array list. For example: "index (6) not in range
      [0..3)" if the function were to be called using the index 6 in a
      size 3 array list. Note the braces and the spacing!


    \item A \code{print} function that takes no parameters and has a \code{void} return
      type.  It prints the array list of characters stored in the
      instance to the screen \code{std::cout}.  It prints the array list like this:
    \begin{lstlisting}
        [CharArrayList of size 5 <<Alice>>]
    \end{lstlisting}

      where, in this example, 5 is the size of the array list and the
      elements are the characters \code{'A', 'l', 'i', 'c', 'e'}.  The
      empty array list would print like this:
    
    \begin{lstlisting}
        [CharArrayList of size 0 <<>>]
    \end{lstlisting}

      Note: There is a single new line after the last \code{]}.  

      Caution:  The output format is essential to get exactly right,
      because your output will be verified automatically.  There is no
      whitespace printed, except the single spaces shown between the
      elements inside the square brackets.  The capitalization must be
      exactly as shown.

    \item A \code{pushAtBack} function that takes an element \code{(char)} and has a
      \code{void} return type.  It inserts the given new element after the end
      of the existing elements of the array list.

    \item A \code{pushAtFront} function that takes an element \code{(char)} and has a
      \code{void} return type.  It inserts the given new element in front of
      the existing elements of the array list.

    \item An \code{insertAt} function that takes an element \code{(char)} and an integer
      index as parameters and has a \code{void} return type.  It inserts the
      new element at the specified index and shifts the existing
      elements as necessary.  The new element is then in the index-th
      position.  If the index is out of range it should throw a C++
      \code{std::range_error} exception with the message ``index (IDX) not in range
      [0..SIZE]" where IDX is the index that was given and SIZE is the
      size of the array list.
      NOTE:  It is allowed to insert at the index after the last element.
      Note that the braces in this message are different from those in
      the \code{elementAt} range error.
      
    \item An \code{insertInOrder} function that takes an element \code{(char)}, inserts
      it into the array list in alphabetical order, and returns nothing.
      When this function is called, it may assume the array list is
      correctly sorted in ascending order, and it should insert the
      element at the first correct index.
      Example 1: Inserting `C' into ``ABDEF" should yield ``ABCDEF"
      Example 2: Inserting `A' into ``ZED" should yield ``AZED."  You can
      rely on the built-in \code{<, >, <=, >=,} and \code{==} operators to compare two
      chars.

    \item A \code{popFromFront} function that takes no parameters and has a \code{void}
      return type.  It removes the first element from the array list.
      If the list is empty it should throw a \code{std::runtime_error}
      exception with the message ``cannot pop from empty ArrayList".

    \item A \code{popFromBack} function that takes no parameters and has a \code{void}
      return type.  It removes the last element from the array list.  If
      the list is empty it should throw an \code{std::runtime_error} exception
      initialized with the string ``cannot pop from empty ArrayList".

    \item A \code{removeAt} function that takes an integer index and has a \code{void}
      return type.  It removes the element at the specified index.  If
      the index is out of range it should throw a \code{std::range_error}
      exception with the message ``index (IDX) not in range [0..SIZE)"
      where IDX is the index that was given and SIZE is the size of the
      array list.

    \item A \code{replaceAt} function that takes an element \code{(char)} and an integer
      index as parameters and has a \code{void} return type.  It replaces the
      element at the specified index with the new element.  If the index
      is out of range it should throw a \code{std::range_error} exception with
      the message ``index (IDX) not in range [0..SIZE)" where IDX is the
      index that was given and SIZE is the size of the array list.

    \item A \code{concatenate} function that takes a pointer to a second
      \code{CharArrayList} and has a \code{void} return type.  It adds a copy of the
      array list pointed to by the parameter value to the end of the
      array list the function was called from.  For example if we
      concatenate \code{CharArrayListOne}, which contains ``cat" with
      \code{CharArrayListTwo}, which contains ``CHESHIRE", \code{CharArrayListOne}
      should contain ``catCHESHIRE".  Note: An empty array list
      concatenated with a second array list is the same as copying the
      second array list.  Concatenating an array list with an empty
      array list doesn't change the array list.  Also an array list can
      be concatenated with itself, e.g concatenating \code{CharArrayListTwo}
      with itself, results in \code{CharArrayListTwo} containing
      ``CHESHIRECHESHIRE".
      
    \item A shrink() function that takes no parameters and has a \code{void} return
      type.  It reduces the object's memory usage to the bare minimum
      required to store its elements (i.e. it does not use any extra
      space).
\end{itemize}

    You may add any private methods and data members.  We particularly
    encourage the use of private member functions that help you produce
    a more modular solution.
    
    Before you start writing any functions please sit down and read this
    assignment specification.  Some of these functions do similar tasks.
    Perhaps it would be prudent to organize and plan your solution using
    the principles of modularity, e. g., helper functions.  This initial
    planning will be extremely helpful down the road when it comes to
    testing and debugging; it also helps the course staff more easily
    understand your work (which can only help).
    
    Also, the order in which we listed the public methods/functions of
    the \code{CharArrayList} class, is not the easiest order to implement them
    in.  If you plan your functions out and identify the easy ones it
    will make your work easier and your final submission better.
    
    If you are having issues planning out your assignment we encourage
    you to come in to office hours as early as possible.
    
\section*{JFFEs (Just For Fun Exercises)}
If you complete the above functions, you may add the following functions. There is no extra credit, but they're fun and educational.

\begin{itemize}
    \item A \code{sort} function that takes no input and has a \code{void} return type.
      It sorts the characters in the list into alphabetical order.
    \item A \code{slice} function that takes a left index and a right index and
      returns a pointer to a new, heap-allocated \code{CharArrayList}.  The
      new array list contains the characters starting at the left
      index and up to, but not including, the right index.  If the
      first index is equal to or greater than the second, it returns a
      new, empty array list.  The left index must be in the range
      [0..SIZE) and the right index must be in the range [0..SIZE]
      where SIZE is the size of the array list. Since the right index
      is not included in the final slice, requesting a slice where the
      right index is 1 index past the last element is still a valid
      request.  If the either index passed is out of range the
      function should throw a \code{std::range_error} with an appropriate
      message.
\end{itemize}

\section*{Implementation Details}
 Copy the starter code from \code{/comp/15m1/files/hw1} to get the exception examples and the example headers and templates of the files below:
 \begin{itemize}
     \item \code{CharArrayList.h}
     \item \code{CharArrayList.cpp}
     \item \code{testCharArrayList.cpp}
 \end{itemize}

 Implement the array list using a dynamically allocated array as
    discussed in class.  You should utilize an \code{expand} or \code{ensureCapacity}
    function that increases/expands your capacity by a reasonable factor
    each time your array list reaches its capacity.
    
    The file \code{CharArrayList.h} will contain your class definition only.
    The file \code{CharArrayList.cpp} will contain your implementation. The
    file\\ \code{testCharArrayList.cpp} will contain the different testing
    functions that you used to unit test (explained below) the different
    parts of your assignment.  We will assess the work in all three files.

    If we don't get to it in class, you should look up how to throw
    exceptions.  Be sure to read \code{simple_exception.cpp} which came with the
    starter code.  That file has a few examples of how an exception can be
    thrown and shows how exceptions impact program execution.  You should
    play around editing the file until you feel comfortable with
    exceptions.  For this assignment, your exceptions should include a
    string with a relevant error message (which we've specified above).
    You'll use the \code{std::runtime_error} or \code{std::range_error} exception type, depending on the circumstance.
    
    You will need to create a program that uses your class to test it.
    Create a file named \code{testCharArrayList.cpp} for this purpose that has
    code testing your functions.  The advice section has more about what
    kinds of things to include here.

    You may create a \code{Makefile} if you like, but it is not required.  Note:
    The Reference page on the course web site links to a document about
    make and \code{Makefile} s as well as a tutorial \code{Makefile} .  You are encouraged to give it a go!
    
\section*{Implementation Advice}

Do NOT implement everything at once!\newline
Do NOT write code immediately!\newline

\noindent Before writing code for any function, draw before and after pictures
    and write, in English, an algorithm for the function.  Only after
    you've tried this on several examples should you think about coding.

First, just define the class, \code{#include} the .h file, define an empty
    main function in your test file, and compile.  This tests whether your
    class definition is syntactically correct.

Then implement just the default constructor.  Add a single variable of
    type \code{CharArrayList} to your test main function, and compile, link, and
    run.

    Then you have some choices.  You could add the destructor next, but
    certainly you should add the print function soon.

    You will add one function, then write code in your test file that
    performs one or more tests for that function.  Write a function, test a
    function, write a function, test function, ...  This is called "unit
    testing."  As you write your functions, consider edge cases that are
    tricky or that your implementation might have trouble with.  You should
    write specific tests for these cases.

    You can organize your testing file as a function (perhaps with
    additional helper functions) to test each item in the interface.  For
    example, you might have a function named \code{test_insertAt} that runs a
    bunch of tests on the \code{insertAt} function.  You can even write tests that
    verify an exception using the C++ "try-catch" statement.  If you do not
    catch an exception, then it will crash your program with an unhandled
    exception.  This is okay, but you should comment out tests that should
    crash with a note to the grader that the exception arose as expected
    (assuming that happened).

    If you need help, TAs will ask about your testing plan and ask to see
    what tests you have written.  They will likely ask you to comment out
    the most recent (failing) tests and ask you to demonstrate your
    previous tests.

    We will evaluate your testing strategy and code for breadth (did you
    test all the functions?) and depth (did you identify all the normal and
    edge cases and test for error conditions?).  Don't write a test a
    function and then delete it!

    For testing exceptions you have two choices - either is fine:
    \begin{itemize}
        \item You may test that exceptions get thrown when appropriate, let it
          crash your test program, and then comment out the test with a
          note that it the exception was thrown (be honest --- we will
          check).
        \item You may write a catch statement in your testing code that catches
          the exception, tests the message, and then prints out a success
          or failure indication.
    \end{itemize}
    
    Be sure your files have header comments, and that those header comments
    include your name, the assignment, the date, and the files purpose. See
    our style guide for more information about commenting your code.
    
\section*{README}
    With your code files you will also submit a \code{README} file, which you
    will create yourself.  The file is named \code{README}.  There is no .text
    or any other suffix.  The contents is in plain text, lines less than
    80 columns wide.  Format your \code{README} however you like, but it should
    be well-organized and readable.  Include the following sections:
    \begin{enumerate}[label=(\Alph*)]
        \item The title of the homework and the author's name (you)
        \item The purpose of the program
        \item Acknowledgements for any help you received
        \item The files that you provided and a short description of what 
           each file is and its purpose
        \item How to compile and run your program
        \item An outline of the data structures and algorithms that you used. 
           Given that this is a data structures class, you always need to
           discuss the data structure that you used and justify why
           you used it.  Specifically for this assignment please
           discuss the features of arrays, major advantages and major
           disadvantages of utilizing an array list as you have in this
           assignment.  The algorithm overview may not be relevant
           depending on the assignment. 
        \item Details and an explanation of how you tested the various parts 
           of assignment and the program as a whole. You may reference 
           the testing files that you submitted to aid in your explanation.
    \end{enumerate}
    Each of the sections should be clearly delineated and begin with a
    section heading describing the content of the section.  It is not
    sufficient to just write ``C" as a section header:  write out the
    section title to help the reader of your file.
    
\section*{Submitting Your Work}
Be sure to read over the style guide before submitting to make sure you comply with all the style requirements. The command is:
\begin{lstlisting}
    provide comp15 hw1 CharArrayList.h \
                       CharArrayList.cpp \
                       testCharArrayList.cpp \
                       README
\end{lstlisting}

\noindent Note:  All the file names can go on one line. The \code{`\'}s are there because we put the file names over multiple lines.  In the Unix shell, a
way to make a command be treated as one line when you type it on multiple lines is to ``escape" the new line by putting a \code{`\'} as the last character before the new line.  You may leave out the \code{`\'} and put all the file names on one terminal line, or you can do the above
\textbf{exactly}, with no space after the \code{`\'}.

\end{document}